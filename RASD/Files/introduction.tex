%This document has been prepared to help you approaching Latex as a formatting tool for your Travlendar+ deliverables. This document suggests you a possible style and format for your deliverables and contains information about basic formatting commands in Latex. A good guide to Latex is available here \href{https://tobi.oetiker.ch/lshort/lshort.pdf}{https://tobi.oetiker.ch/lshort/lshort.pdf}, but you can find many other good references on the web. 

%Writing in Latex means writing textual files having a \texttt{.tex} extension and exploiting the Latex markup commands for formatting purposes. Your files then need to be compiled using the Latex compiler. Similarly to programming languages, you can find many editors that help you writing and compiling your latex code. Here \href{https://beebom.com/best-latex-editors/}{https://beebom.com/best-latex-editors/} you have a short oviewview of some of them. Feel free to choose the one you like.  

%Include a subsection for each of the following items\footnote{By the way, what follows is the structure of an itemized list in Latex.}:

% -----------------------------------------------------

\subsection{Purpose}
In today's job market, when looking for a job, having a previous work experience has become an important criteria of evaluation, therefore more and more students try to find an internship. On the other side, many companies are willing to invest into young talents in order to build a skilled workforce that can help them to drive innovation and foster and secure a competitive edge in the future. \\
Student\&Companies is a platform that offers to students the possibility to look up for internships and to companies the possibility of advertising. The matchmaking process of student and companies is enabled through a (proprietary) recommendation algorithm, that creates the most suited matches, through the collection of statistics and feedback. \\
Moreover Students\&Companies allows for the management of the selection process and the monitoring of the execution of the internships. The goal is to provide a platform for both students and companies to make the process of finding, offering and tracking internships easier and more efficient for both parties.

% ----

\subsubsection{Goals}
\begin{enumerate}[label={[G\arabic*]}]
    \item \textbf{Internship Lookup for Students}: 
    Help students in their search for an internship by connecting their profiles with well suited companies internship offers.
    \item \textbf{Internship Advertisement for Companies}: Enable companies to advertise their internships to students and inform them about availability of eligible candidates.
    \item \textbf{Selection Process Management}: Support the interaction and selection by providing a platform to companies to set up and conduct interviews, gather structured information about students and finalize the selections.
    \item \textbf{Data Collection for Recommendation System}: Collect statistics in order to allow the efficiency of the matchmaking process of the recommendation system 
    \item \textbf{Enhance Communication}: Enhance communication between students and companies, through a shared space where they can exchange information, raise problems and collect complaints about the internships
\end{enumerate}

% -----------------------------------------------------

\subsection{Scope}
The platform S\&C has two major stakeholders: students and companies. On one hand, students upload their CV and indicate their skills and domain of interest on the platform. Then they can proactively look out for internships or they can get suggestions from the platform, when new positions become available. On the other hand, companies can open internship positions, with the related domain and required skills, advertise those positions but also get notified when profiles of students, aligned with the requirements of one of their open positions, appear on the platform.
Once both parties express interest on the platform, a contact between the two is established. Then the management of the set up and the conduct of interview starts. When a candidate is selected, the platform tracks the conduction of the internship through feedback, complaints and exchange of information by both parties. Indeed a specific space of the website is devoted to communication between a student and company that have started an internship process together. The system run analytics on statistics collected by the platform itself on both students and companies to guarantee the best match possible between the two.

% ----

\subsubsection{World phenomena}
\begin{enumerate}[label={[WP\arabic*]}]
\item {Companies need interns for a position}
\item {Students write their own CVs}
\item {Students acquire new skills }
\item {Companies and students carry out internships on which they agreed upon}
\end{enumerate}

% ----

\subsubsection{Shared phenomena}
\begin{itemize}
    \item World controlled
    \begin{enumerate}[label={[SPWC\arabic*]}]
    
    \item {Companies create an internship position}
    \item {Companies carry out the selection process}
    \item {Companies select one student among the candidates}
    \item {Companies write complaints}
    \item {Companies communicate problems}
    \item {Students fill their profile with their personal information}
    \item {Students look out for an internship}
    \item {Students undertake assessments to test their skills on the platform}
    \item {Students write complaints}
    \item {Students communicate problems}
    
    
    \end{enumerate}
    \item Machine controlled
    \begin{enumerate}[label={[SPMC\arabic*]}]
    \item {The system notifies students of a newly available internship}
    \item {The system notifies the company of a newly available student that fits an internship role}
    \item {Contact between a student and a company is established} % I'm not sure it's Machine controlled
    \item {Companies provide feedback} % prompted by the system
    \item {Students provide feedback} % prompted by the system
    
    \end{enumerate}

\end{itemize}

% -----------------------------------------------------

\subsection{Definitions, Acronyms, Abbreviations}

\subsubsection{Definitions}
\begin{itemize}
    \item 
\end{itemize}

% ----

\subsubsection{Acronyms}
\begin{itemize}
    \item {S\&C: Students \& Companies}
    \item {UI: User Interface}
    \item {UML: Unified Modeling Language}
    \item {RASD: Requirement Analysis and Specification Document}
    \item {DD: Design Document}
\end{itemize}

% ----

\subsubsection{Abbreviations}
\begin{itemize}
    \item {G*: Goal}
    \item {WP*: World Phenomena}
    \item {SPWC*: Shared Phenomena World Controlled}
    \item {SPMC*: Shared Phenomena Machine Controlled}
    \item {FR*: Functional Requirement}
    \item {UC*: Use Case}
\end{itemize}

% -----------------------------------------------------

\subsection{Revision History}

% -----------------------------------------------------

\subsection{Reference Documents}
This document is based on the following reference documents:
\begin{itemize}
\item {The document of the 2024/2025 assignment for RASD and DD }
\item {The slides of the course found on WeBeep}
\end{itemize}

% -----------------------------------------------------

\subsection{Document Structure}
\begin{enumerate}
\item\textbf{Introduction}\\
The introduction identifies the purpose and the scope of the project, defines the goals, lists the related world and shared phenomena, explains acronyms, abbreviations that will be used in the document, specifies the version of the document and the reference document on which the current version it is based.
\item\textbf{Overall Description}\\
The overall description points out and explains the various scenarios, including class diagrams for the most difficult ones. It also includes product functions, user characteristics plus assumptions for the specific domain of the application, dependencies of the applications and constraints.
\item\textbf{Specific Requirements}\\
The specific requirements section defines the different interface and most importantly outlines all the functional requirements, alongside with all the possible use cases.
\item\textbf{Formal Analysis} \\
The part about formal analysis uses Alloy to formally verify the most difficult parts of the project
\item\textbf{Effort Spent}\\
The effort spent is described by a simple table
\item\textbf{References}\\
It has pointers to all the resources used or cited in this project
\end{enumerate}



%what you write here is a comment that is not shown in the final text